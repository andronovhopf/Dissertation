\documentclass[../uwthesis.tex]{subfiles}
\begin{document}

\chapter {Concluding remarks}
In the course of four experiments, my coauthors and I have investigated several contemporary theories of dyslexia using psychophysical methods. Considering our results in the context of the literature as a whole, we arrive at several conclusions that we hope will prove useful to the field. 

\emph{First, psychometric function shape on the phoneme categorization task is a robust and replicable predictor of reading skill.} This relationship is not particular to a type of phonetic cue (purely spectral or spectrotemporal), cue duration, frequency distribution of stimulus presentations (uniform or bimodal), or the precise format of the psychophysical experiment design (3-interval or single interval). The result is not easily accounted for by considering the influence of attentional lapses, recent stimulus presentations, or engagement during the task. 

Our results are both reassuring and problematic for historical interpretations of what phoneme categorization means about language and reading. On the one hand, our results encourage the view that a great deal of literature on phoneme categorization in struggling readers reflects a very real linguistic difference \citep{Mody1997,Serniclaes2004,Noordenbos2015}. Because this trend can be seen in so many experimental contexts---including synthetic and natural stimuli, purely spectral and purely temporal contrasts, vowels and consonants, in children and adults, with tonal and atonal language speakers \citep{Noordenbos2015}---we believe it is no longer tenable to suggest the origin is a particular sensory deficit, as in any flavor of the temporal processing hypotheses of dyslexia. While our results alone do not rule out that sensory deficits of any kind could be present in some individuals with dyslexia, we conclude that the relationship between phoneme categorization and reading skill is not attributable to such a deficit in most cases. 

At the same time, this work does not strongly support the classical thinking that phoneme categorization is causally related to phonological awareness: throughout the literature, it is common to see phoneme categorization equated with the ability to segment words into phonemes or blend phonemes into words. In other words, it is assumed that the relationship between phoneme categorization and reading skill is mediated by phonological awareness. Yet we find only limited empirical evidence for this model, as do several other recent studies in relatively large samples \citep{Robertson2009,Hakvoort2016,Snowling2019LongitudinalDyslexia}. Our work adds to accumulating evidence that phoneme categorization is a predictor of reading skill separately from phonological awareness. This is perhaps unsurprising considering that only limited empirical evidence supports a view of speech perception in which context-invariant phonemes are extracted from speech in natural listening conditions. To quote Holt and Lotto (\citep{Holt2010SpeechCategorization} p. 1223):
\begin{quote}
    "Everyday speech perception “in the wild” is likely to tap into a broader set of processes than those captured in individual laboratory tasks. It is important to note that this is not to suggest that adult (or even infant or animal) listeners cannot categorize speech; there is abundant evidence that they can. Rather, these data suggest that the cognitive and perceptual processes involved in speech categorization and those in online perception of fluent speech may not be one and the same."
\end{quote}

Indeed, it is perhaps more important for development that children learn to be appropriately \emph{un}-categorical as listeners: the task of speech perception involves adapting to many different talkers with many different dialects. The theoretical foundation linking phoneme categorization to phoneme awareness in dyslexia seems to presuppose that speech perception proceeds through a necessary downsampling stage from rich acoustic cues to abstract phonemes, which are then mapped to graphemes during literacy training---an idea that's been substantially challenged in the decades since the study of phoneme categorization in dyslexia began \citep{Cleary2001,Port2005,Port2007,McMurray2009,Toscano2010}. Without disregarding the immense contributions of these classical ideas to promoting many thoughtful avenues of research, it may now be time to reconsider the theoretical framework connecting speech to literacy. 

This relates to a second conclusion of the dissertation: \emph{a purely phonological account of dyslexia is insufficient to account for most cases of disabled and typical reading}. Not only is phonological awareness itself inadequate as a predictor, we find evidence for several other predictors that do not appear to be mediated by phonological awareness. Automaticity, phonological awareness, visual motion processing and non-sensory aspects of decision making appear to each confer separate risk for reading disability (though it is important to note that our results do not imply that each risk factor has a causal relationship with reading skill, as the predictors we identified may only be correlates of an unknown causal mechanism). Our results add to the emerging picture that a single-deficit model, or even double-deficit model, of reading disability is over-simplified \citep{Pennington2012IndividualModels.,Peterson2015DevelopmentalDyslexia,Ziegler2019ModelingDyslexia}. While we do not contest that phonological awareness remains an important predictor of reading ability, our findings are at odds with the notion of a characteristic "profile" of cognitive impairments in dyslexia.

In light of our results and the broader literature, we can make several proposals for future research. The shift we hope to see in dyslexia research is less emphasis on "core deficits" and more on "risk factors"---i.e., a transition from a deterministic to stochastic view acknowledging that dyslexia, and likely other learning disabilities, may not have a unifying causal mechanism that explains the majority of cases. Taking this view will require less reliance on small-scale psychophysical studies comparing groups of "control" and "dyslexic" participants and increased focus on studies with hundreds, or even thousands, of participants where natural variability in reading skill can be exploited. 

Several experimental design considerations can be recommended on the basis of our work:
\begin{itemize}
    \item \textbf{Browser-based psychophysical batteries to more clearly assess the relationship of sensory processing to language development.} Today, many visual and auditory psychophysical tasks can be delivered over the internet to vastly increase the number and diversity of participants. Through partnerships with schools, school-aged children could be assessed on tests of sensory processing (such as spectrotemporal modulation detection or sound localization) as well as speech perception and language skills to better understand the strength of relationships between these various levels of processing. At this point, it is not clear to what extent individual variability in basic sensory processes constrain language learning, and large-scale studies can help establish bounds on the magnitude of this effect. 
    \item \textbf{More targeted studies of how naturalistic speech perception relates to reading skill.} While categorical phoneme labeling is a predictor of reading skill, its relationship to everyday speech perception remains unclear. Methods such as a eyetracking and pupillometry allow experimenters to take nuanced time-series measures of listening effort and on-line prediction in response to natural speech. This would allow researchers to directly test many hypotheses about dyslexia, including whether struggling readers engage in less prediction of upcoming words (by the statistical learning hypothesis) or exert more effort than their peers in difficult listening conditions (sensory theories). 
    \item \textbf{Literacy intervention studies to examine how reading changes perception.} Some researchers have speculated that becoming literate drives top-down changes in speech perception \citep{Huettig2019LiteracyLanguage}, and vocabulary growth (often a side-effect of literacy) may drive changes in sensitivity to phonetic cues \citep{Cleary2001}. Intervention studies in which children receive literacy training and are tested on speech perception or auditory measures before and after can directly test these ideas. Studies of this nature will help settle debates about causes of reading disability versus consequences of it.
    \item \textbf{Going beyond group-level analyses.} Because our results and others point to a multi-dimensional model of reading skill without clear discontinuities or clusters of readers, we suggest that studies that seek to assess population-level risk factors of reading disability prioritize diverse recruitment and rigorous statistical modeling over group comparisons between typical readers and "pure dyslexic" individuals without any other known diagnoses. In a probabilistic framework, understanding what protective mechanisms prevent children from going on to develop a learning disability is equally as important as understanding the risk factors. There will also be much to gain by considering similarities between children diagnosed with dyslexia and other developmental disorders that are co-morbid with it, such as ADHD and dyscalculia. 
    
\end{itemize}

There is still much to be done to explain the variability in reading skill that occurs in the population, and more importantly, to provide timely and effective services to all learners who require them. As we learn to better characterize the prevalence of various risk factors, it may be possible---with institutional support and funding---to make early preventative literacy training for at-risk children a standard part of education. I am hopeful for this future if we as a society choose to prioritize the social good of literacy for all children. 













\end{document}